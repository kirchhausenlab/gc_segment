\documentclass{article}

\usepackage{graphicx}
\usepackage{amsmath}
\usepackage{url}
\usepackage{listings}
\usepackage{color}
\usepackage{hyperref}

\definecolor{mygreen}{rgb}{0,0.6,0}
\definecolor{mygray}{rgb}{0.5,0.5,0.5}
\definecolor{mymauve}{rgb}{0.58,0,0.82}

\lstset{ 
  backgroundcolor=\color{white},   % choose the background color; you must add \usepackage{color} or \usepackage{xcolor}; should come as last argument
  basicstyle=\footnotesize\ttfamily, % the size of the fonts that are used for the code
  breakatwhitespace=false,         % sets if automatic breaks should only happen at whitespace
  breaklines=true,                 % sets automatic line breaking
  captionpos=b,                    % sets the caption-position to bottom
  commentstyle=\color{mygreen},    % comment style
  deletekeywords={...},            % if you want to delete keywords from the given language
  escapeinside={\%*}{*)},          % if you want to add LaTeX within your code
  extendedchars=true,              % lets you use non-ASCII characters; for 8-bits encodings only, does not work with UTF-8
  firstnumber=1,                   % start line enumeration with line 1000
  frame=single,                    % adds a frame around the code
  keepspaces=true,                 % keeps spaces in text, useful for keeping indentation of code (possibly needs columns=flexible)
  keywordstyle=\color{blue},       % keyword style
  language=Bash,                   % the language of the code
  morekeywords={*,...},            % if you want to add more keywords to the set
  numbers=left,                    % where to put the line-numbers; possible values are (none, left, right)
  numbersep=5pt,                   % how far the line-numbers are from the code
  numberstyle=\tiny\color{mygray}, % the style that is used for the line-numbers
  rulecolor=\color{black},         % if not set, the frame-color may be changed on line-breaks within not-black text (e.g. comments (green here))
  showspaces=false,                % show spaces everywhere adding particular underscores; it overrides 'showstringspaces'
  showstringspaces=false,          % underline spaces within strings only
  showtabs=false,                  % show tabs within strings adding particular underscores
  stepnumber=1,                    % the step between two line-numbers. If it's 1, each line will be numbered
  stringstyle=\color{mymauve},     % string literal style
  tabsize=2,                       % sets default tabsize to 2 spaces
  title=\lstname                   % show the filename of files included with \lstinputlisting; also try caption instead of title
}

\newcommand\dataexp{\path{data1expansion/FIBSEM_datasync3}}
\newcommand\scratch{\path{/media/scratch/mihir/}}
\newcommand\guipath{\path{/home/tkadmin/mihir/gui_annotator/}}
\newcommand\volpy{\path{VolumeAnnotator.py}}
\newcommand\segpy{\path{segment.py}}
\newcommand\cfg{\path{configs/}}
\newcommand\out{\path{output/}}
\newcommand\py{\texttt{python}}
\newcommand\env{\texttt{cellseg}}

\title{\bfseries GUI Annotator: Documentation}
\author{Kirchhausen Lab \\
\url{mihir.sahasrabudhe@childrens.harvard.edu}}
\date{\today{}}

\begin{document}
    \maketitle
    \hrule

    \section{Introduction}
        This short document notes some essentials of the GUI annotator. 

    \section{GUI Annotator}
        The code for the GUI annotator is in \guipath . The specific
        file concering the annotator is inside this directory, and is 
        called \volpy . There are other \py\ files inside this directory
        but they can be ignored for the purposes of this documentation. 

    \section{The Virtual Environment}
        To use the GUI annotator, it has to be launched from the terminal.
        To open the terminal, one can either run the command 
        \texttt{gnome-terminal} after pressing \texttt{Alt+F2}, or one 
        can click on the terminal icon on the sidebar on the left of the 
        screen. 

        Once a terminal window is open, change the directory to \guipath\
        using the following command.
\begin{lstlisting}
cd %*\guipath*)
\end{lstlisting}

        The next step is to activate the virtual environment \env . The virtual
        environment has all the necessary libraries installed, so the 
        code will not work without activating it. To activate the virtual 
        environment, use the following command. 
\begin{lstlisting}
source activate %*\env*)
\end{lstlisting}

        This activates the environment \env . Finally, you should be able to 
        see a prompt like the following.
\begin{lstlisting}
(%*\env*))tkadmin@tkhpc32 %*\guipath*)$
\end{lstlisting}

    \section{Configuration Files}
        Experiments are defined by configuration files, which define the 
        \emph{parameters} to be used. It is preferable to have a separate
        experiment, and hence configuration file, for every separate annotation. 
        Each configuration file specifies a few options. Further, the configuration 
        file is shared between the annotation and the segmentation tools. However,
        for the purposes of this documentation, we limit our scope to the 
        annotation only. 

        The annotation options that can be specified in a configuration file are
        \begin{enumerate}
            \item \texttt{data\_path}: This option specifies the absolute path to the 
            directory which contains all slices from a particular acquisition. 
            This data are not cropped. Hence, for configuration files which 
            focues different organelles in the same cell, this option can have
            the same value. 

            \item \texttt{coords}: This option specifies the $X$- and $Y$- coordinates
            of the bounding box around the organelle to be segmented. It also
            specifies the width $W$, and the height $H$ (along $X$ and $Y$, respectively)
            for this bounding box. 
        \end{enumerate}
        

    \section{Data}
        The data used are images from the FIB-SEM, which can be found
        in the directory \dataexp . We can copy whatever data are
        needed


\end{document}
